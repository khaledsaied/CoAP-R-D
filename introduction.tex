\section{Introduction}
% 
% Some journals put the first two words in caps:
% \IEEEPARstart{T}{his demo} file is ....
% 
% Here we have the typical use of a "T" for an initial drop letter
% and "HIS" in caps to complete the first word.
\IEEEPARstart{C}{\MakeLowercase{onstrained}} devices are an essential part of Internet of Things (IoT) and therefore there is focus on developing lightweight protocols which enables different connectivity models for constrained devices.  
CoAP is one of the emerging lightweight application protocol which is designed for machine-to-machine (M2M) applications for use in constrained environments.   

Previously a research has been done in the IoTivity framework which operates as a middleware trying to solve interoperability issues in the IoT \cite{interoperabilityChallenge}. Results from the research shows that IoTivity does not have the capability to communicate with the cloud in an appropriate way. 

IoTivity made use of the CoAP protocol as the only application protocol. 
A new version of the IoTivity framework has been published recently, where they have implemented a solution to connect IoT devices to the Cloud \cite{iotivity1.1}. The solution was based on a newer draft for CoAP \cite{coapTCP} which presents a modified CoAP that runs over TCP instead of UDP \cite{coapUDP}.
The TCP protocol is more complex compared with the UDP protocol and thus it is not easy to employ on constrained devices. In this paper it will be investigated if it is advantageous to use TCP as a transport protocol in CoAP instead of UDP. This investigation will be done by analysing the two versions of the CoAP protocol using a qualitative and quantitative analytic method.


This research will  for implementing a more complex transport protocol on the lightweight application protocol CoAP which already has UDP implemented.

This research will focus on 

%BROKER
Engineering Task Force (IETF) has published a draft in July 2014 presenting a publish-subscribe broker that can be used between CoAP clients. An advantage with the publish-subscribe paradigm is that the constrained devices can avoid to function as servers and can thereby sleep the most of time.

CoAP is a reasonable choice for Device-to-Device communication,  but it is also a possible choice for Device-to-Cloud communication though it has some limitations which are handled differently by the three CoAP setups.

The purpose of this paper is to find the most optimal way for connecting constrained devices to the Cloud by investigating and comparing the three different presented variations of the CoAP protocol. The rest of this paper is structured as follows.  Section 2 presents... . Section 3 ... . Section 4 discusses the results ... . Section 5 concludes the paper.     
