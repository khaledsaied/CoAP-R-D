% As a general rule, do not put math, special symbols or citations
% in the abstract or keywords.
\begin{abstract}
Internet of Things (IoT) is expected to consist of billions of devices by 2020. Constrained devices are an essential part of IoT and therefore the focus is on developing lightweight protocols enabling, among others, cloud communication. CoAP, a lightweight web transfer protocol, is developed in recent years and is already in use. A new version of CoAP has been proposed which presents a modified CoAP that runs over TCP instead of UDP. IoTivity has used CoAP before and has recently published a new version with CoAP over TCP implemented, specifically for cloud communication. In this paper a comparison between CoAP and CoAP over TCP is made focusing on the energy consumption by analysing the bandwidth and latency. A concrete case is given with the use of Arduino Due using an Arduino WiFi Shield to exemplify the energy consumption of the two versions of CoAP in a constrained device.
The evaluation of the results shows ???.
\end{abstract}

% Note that keywords are not normally used for peerreview papers.
\begin{IEEEkeywords}
CoAP, CoAP over TCP, IoTivity, constrained devices, IoT, cloud, Arduino, UDP, TCP.
\end{IEEEkeywords}
