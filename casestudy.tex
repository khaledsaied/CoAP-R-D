\section{Case Study} \label{sec:casestudy}
To better understand the results and which effect they can have on constrained devices, a concrete case is given with use of the Arduino WiFi Shield. This case will exemplify how much energy and memory the two CoAP versions will consume if used in an Arduino device.

\subsection{Memory}
For measuring the memory usage, an IoTivity CoAP implementation for Arduino is used. IoTivity currently does not support a CoAP over TCP implementation for Arduino and therefore the memory calculations are only made for a IoTivity CoAP implementation running on an Arduino Due.
 
%The memory consumption can only be made on the regular CoAP version as IoT is based on the implementation of IoTivity client applications 
The memory consumption are presented in table \ref{tab:memory}:
%TABLE
\begin{table}[bht]
	\renewcommand{\arraystretch}{1.3}
	\caption{Memory Consumption in Arduino}
	\label{tab:memory}
	\centering
	\begin{tabular}{|c|c|c|}
		\hline
		\bfseries  & \bfseries IoTivity App (CoAP) & \bfseries IoTivity App (CoAP over TCP) \\
		\hline
		\textbf{RAM} & 3 kB & - \\
		\hline
		\textbf{Flash} & 95.18 kB & - \\
		\hline
	\end{tabular}
\end{table}

The RAM measurements are done by using an AVR-tool called avr-size. The tool measures only the size of the initialized variables \cite{Check55:online}.

\subsection{Energy}
The bandwidth measurements can easily be converted to the amount of energy consumption for a specific device. The calculations are shown below:

%calculation...
%\textbf{En god ide er at maale energiforbruge / beregne det i joule. Giver et konkret svar om den kan bruges i specifikke typer af constraint devices}

%Energy calculation:
%watt = joule per second
%802.11.g <-- speed 
%590 mW - transmit
%230 mW - receive