\section{Case Study} \label{sec:casestudy}
To better understand the results and which effect they can have on constrained devices, a concrete case is given with use of the Arduino Due with an Arduino WiFi Shield. This case will exemplify how much memory and energy the two CoAP versions will consume if used in the mentioned device.

\subsection{Memory}
For measuring the memory usage, an IoTivity CoAP implementation for Arduino is used. IoTivity currently does not support CoAP over TCP for Arduino and therefore the memory measurements are only made for an IoTivity CoAP implementation running on an Arduino Due.
 
%The memory consumption can only be made on the regular CoAP version as IoT is based on the implementation of IoTivity client applications 
The memory consumption are presented in table \ref{tab:memory}:
%TABLE
\begin{table}[bht]
	\renewcommand{\arraystretch}{1.3}
	\caption{Memory Consumption in Arduino}
	\label{tab:memory}
	\centering
	\begin{tabular}{|c|c|c|}
		\hline
		\bfseries  & \bfseries IoTivity App (CoAP) & \bfseries IoTivity App (CoAP over TCP) \\
		\hline
		\textbf{RAM} & 3 kB & - kB \\
		\hline
		\textbf{Flash} & 95.18 kB & - kB \\
		\hline
	\end{tabular}
\end{table}

The RAM measurements are done by using an AVR-tool called avr-size. The tool measures only the size of the initialized variables \cite{Check55:online}.

\subsection{Energy}
For calculating the amount of energy that will be consumed if the aforementioned chosen devices was running the complete communication scenario experiment, different factors has to be known. These are the energy consumption of the WiFi component (HDG204) in the WiFi shield, both for transmitting and receiving data, and the data rate of the WiFi component. These data are read from the datasheet for the HDG2014 component \cite{hdg204:datasheet}. The used WiFi component supports the standard 802.11.g which provides a rate up to 54 Mbit/s. It is however assumed that the rate for the transmissions is always 54 Mbit/s even though it can fluctuate. 
It is assumed that the WiFi shield is in sleep-mode every time it is idle. Furthermore, worst case is assumed for the latency of the transactions. This gives a latency of 2.7 s for CON-CoAP and 0.4 s for CoAP over TCP.


The calculations for the energy consumption for CON-CoAP are:
\begin{equation}\label{eq:txcon}
TxEnergyCon = \frac{4920 byte}{7077888 byte/s } \cdot 590 mJ/s = 0.41 mJ   
\end{equation}
\begin{equation}\label{eq:rxcon}
RxEnergyCon = (60 \cdot 2.7 s - \frac{4920 byte}{7077888 byte/s }) \cdot 230 mJ/s = 37259.84 mJ   
\end{equation}
%\begin{equation}\label{eq:rxcon}
%RxEnergyCon = \frac{4140 byte}{7077888 byte/s } \cdot 230 mJ/s = 0.135 mJ   
%\end{equation}
\begin{equation}\label{eq:idlecon}
IdleEnergyCon = (1800 s - 60 \cdot 2.7 s) \cdot 0.4 mJ/s = 655.2 mJ   
\end{equation}
\begin{equation}\label{eq:totalcon}
TotalEnergyCon = 0.41 mJ + 37259.84 mJ +655.2 mJ = 37915.45 mJ %0.545 mJ 
\end{equation}

Equation \ref{eq:txcon} and \ref{eq:rxcon} are the energy consumptions for respectively transmitting and receiving of data. Equation \ref{eq:idlecon} is the energy consumptions for being in idle state.
Equation \ref{eq:totalcon} is the total energy consumption for the complete communication scenario.

The same calculations are also made for CoAP over TCP:
\begin{equation}\label{eq:txtcp}
TxEnergyCon = \frac{6310 byte}{7077888 byte/s } \cdot 590 J/s = 0.526 mJ   
\end{equation}
\begin{equation}\label{eq:rxtcp}
RxEnergyCon = (60 \cdot 0.4 s - \frac{6310 byte}{7077888 byte/s }) \cdot 230 J/s = 5519.79 mJ   
\end{equation}
%\begin{equation}\label{eq:rxtcp}
%RxEnergyCon = \frac{4232 byte}{7077888 byte/s } \cdot 230 J/s = 0.138 mJ   
%\end{equation}
\begin{equation}\label{eq:idletcp}
IdleEnergyCon = (1800 s - 60 \cdot 0.4 s) \cdot 0.4 mJ/s = 710.4 mJ %655.2 mJ   
\end{equation}
\begin{equation}\label{eq:totaltcp}
TotalEnergyCon = 0.526 mJ + 5519.79 mJ + 710.4 mJ = 6230.72 mJ 
\end{equation}

An Arduino Due powered with a 9V Alkaline battery running the complete communication scenario with CON-CoAP repeatedly has a lifetime of approximately 7.38 hours, as calculated: 
\begin{equation}\label{eq:concoaphours}
Lifetime = \frac{17820000 mJ }{(2 \cdot 37915.45 mJ)+2340000} =7.38 hours %7.6153811 
\end{equation}
%\begin{equation}\label{eq:years}
%Lifetime = \frac{17820000 mJ }{(2 \cdot 0.545 mJ)} \cdot \frac{1}{24} \cdot \frac{1}{365} = 1,866.28 years
%\end{equation}

For CoAP over TCP the lifetime is approximately 7.58 hours, as calculated:
\begin{equation}\label{eq:years}
Lifetime = \frac{17820000 mJ }{(2 \cdot 6230.72 mJ)+2340000 } = 7.58 hours %7.6153803 
\end{equation}
%\begin{equation}\label{eq:years}
%Lifetime = \frac{17820000 mJ }{(2 \cdot 0.664 mJ)+2340000 } \cdot \frac{1}{24} \cdot \frac{1}{365} = 1,531.81 years
%\end{equation}

Notice that the most of the energy consumption is used by the Arduino Due itself and not by the WiFi shield and this is because it is assumed that Arduino Due does not use any power saving.
% (17820000/((2*0,664)+2340000 mJ))

%bandwidth size (bytes)/data rate (bytes/s)  => time for  sending this packet
%time for  sending this packet * transmit energy consumption
%(6310/7077888)*590 = 0,5259902389 milli Joule
%time for  sending this packet * receive energy consumption
%(4232/7077888)*230 = 0,1375212493 milli Joule
%Energy calculation:
%watt = joule per second
%802.11.g <-- 54Mbit/per second (wireless rate) = 7077888 bytes/per second 
%590 mW - transmit energy consumption
%230 mW - receive energy consumption
%calculation...
%\textbf{En god ide er at maale energiforbruge / beregne det i joule. Giver et konkret svar om den kan bruges i specifikke typer af constraint devices}

