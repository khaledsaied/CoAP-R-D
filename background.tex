\section{Background}




\subsection{Middleware as Solution for Interoperability}
Middleware has proven to be a solution in many environments for ensuring interoperability within distributed systems. In IoT the same way of thinking has been adapted. 
Several projects in collaboration between big companies are currently developing  middleware solutions for IoT and one of these solutions is IoTivity which supports CoAP as an application protocol.
 
In \cite{interoperabilityChallenge} the IoTivity framework is evaluated according to defined requirements among others the connectivity where different connectivity models are evaluated. These connectivity models are device-to-device, device-to-gateway, and device-to-cloud. Results from the evaluation of IoTivity shows that device-to-cloud communication within constrained devices is challenging.

\subsection{Device-to-Cloud}
Existing solutions for communicating with device-to-cloud are HTTP, MQTT, CoAP...etc.

compare solutions....


HTTP 

MQTT



\subsection{Constrained Application Protocol (CoAP)}
CoAP is a lightweight web transfer protocol specialized for use with constrained devices. CoAP is designed for M2M communication which includes the device-to-cloud connectivity model.
CoAP runs over the UDP transport protocol which helps keeping the message overhead small and thereby limits the need for fragmentation.
CoAP can be reliable even though UDP is used as transport protocol because CoAP handles the reliability and congestion control by enabling the message type Confirmable (CON).
 
For security the Datagram Transport Layer Security (DTLS) which is based on the stream-oriented Transport Layer Security (TLS) is used.
 
--UDP can only transfer small amounts of data

\subsubsection{CoAP over TCP}
CoAP over TCP is an active internet draft made by The Internet Engineering Task Force (IETF) \cite{IETF}. 
In this draft it is proposed to use TCP as a transport protocol instead of UDP.

TLS
   
TCP has less risk of being blocked by firewalls than UDP
No lost packages/data
Provides additional information about session life to NAT
Long lifetime
No need for handling reliability
(Exchanges messages asynchronously even though it runs TCP)
TLS
Is the changes in the header compatible with the current CoAP implementation?


Background: Beskrivelse af de to protokoller, forholdsvis kort (undlade detaljer, som ikke er relevant for artiklen)


